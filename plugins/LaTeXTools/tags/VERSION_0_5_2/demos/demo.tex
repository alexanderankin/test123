\documentclass[12pt,a4paper]{report}
\newtheorem{definition}{Definition}[chapter]
\begin{document}
	\chapter{Intro}
		Hello dear reader. This is a non-sense demo file to show the capabillities
		of the \LaTeX-Tools plugin for jEdit.
		\section{What is this}
			So what the heck is this exactly?
			\begin{definition}[\LaTeX-Tools]
				\LaTeX-Tools are a plugin for jEdit.
			\end{definition}
	
	\chapter{Basics}
		\LaTeX-Tools help you to navigate around in you \LaTeX source. 
		\begin{figure}[ht]
			\centering
			\includegraphics*[width=7cm]{picture}
			\caption{example picture}
			\label{figure:picture}
		\end{figure}

		\section{Functional overview}
			\LaTeX-Tools offer many features:
			\begin{itemize}
				\item{navigator}
				\item{reference completion}
				\item{BibTeX completion}
			\end{itemize}
			The navigator understands several environments and commands:
			\begin{table}[h]
				\begin{tabular*}{\textwidth}[h]{@{\extracolsep{\fill}}|r|l|}
					\hline
						\textbf{item}        & \textbf{description}\\
					\hline
						table, tabular       & environment for tables
				\end{tabular*}
				\caption{environments and commands}
				\label{table:envandcmd}
			\end{table}
		\section{Example tex file}
			\begin{verbatim}
				bla bla
			\end{verbatim}

	\chapter{Appendix}
		And another chapter.
		\section{Greeting}
			Thanks for anyone who help us with the development.
			\subsection{Particular greetings}
				Big thanks to Slava, for the fantastic jEdit.

\end{document}
