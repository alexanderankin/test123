\documentclass[a4paper,10pt,twoside]{article}
\pagestyle{headings}
\usepackage{a4wide}
\usepackage[english]{babel}
\usepackage[colorlinks,hyperfigures,backref,bookmarks,draft=false]{hyperref}
\usepackage{pgf}
\usepackage{subfigure}  % use for side-by-side figures


\title{A sample latex document}

\begin{document}

\maketitle

\begin{abstract}
Aspell has the ability to ignore some parts of a text file when spell-checking.
For instance, in this \latex{} document, the \emph{page style} commands won't
get a warning, whereas the plain occurrence would (hence the space).
Here is a mispelled word...
\end{abstract}

\section{Supported languages}
Aspell 0.60 can handle 
\begin{itemize}
\item c++ and the like (java, c, \dots) : it will only check comments (between {\ttfamily /**/})
	  and string literals.
\item to be continued...
\end{itemize}
\end{document}
